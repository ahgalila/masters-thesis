\chapter{Conclusion} \label{sec:findings-and-conclusion}

In Chapter 1 we described the challenges that an individual human would have in learning to survive alone in the jungle. She or he would have to observe a large number of examples of a concept in order to learn the concept. We discussed how by using shared knowledge from other humans and relying on their past experiences, the lone learner is able to overcome learning from impoverished datasets. In traditional machine learning, computer systems do not have the benefit of shared knowledge and past experiences. They must rely on sufficient numbers of training examples to achieve the same level of accuracy as a human would.

In Chapter 3 we hypothesized that in the same manner that human learners are able to use shared knowledge to overcome the challenges they encounter while learning, machine learning systems can overcome impoverished datasets using shared knowledge. Our theory states that by introducing a clear and concise symbol channel, that is analogous to the knowledge shared by human learners, neural networks can overcome learning difficulties caused by small training sets and produce more accurate models. In Chapter 4, we presented a series of experiments that show how this theory can be applied to training recurrent neural networks to perform basic arithmetic on images of handwritten digits.

This final chapter starts by providing a summary of our findings and reflects on the objectives of our research (Section \ref{sec:findings-and-conclusion-summary-of-findings}). Next we describe the contributions made by our work (Section \ref{sec:findings-and-conclusion-research-contributions}). Finally, a list of suggested future work is presented in Section \ref{sec:findings-and-conclusion-future-work}.

\section{Summary of Findings} \label{sec:findings-and-conclusion-summary-of-findings}

Section \ref{sec:introduction-research-objective-scope-research-objective} stated two objectives for our research. The first objective was to validate the hypothesis by showing that neural networks trained using symbols perform significantly better than those trained without the presence of symbols. We started by developing several recurrent neural networks to perform arithmetic on images of noisy handwritten digits with the aid of symbols represented as one-hot vectors. Two methods of presenting the symbols to the networks were developed and tested. The first used a separate input channel alongside the noisy inputs. The other method trained the recurrent network to classify the incoming digits on the first time steps. The results obtained by these initial models confirmed our hypothesis, and therefore accomplished the first objective, by showing that regardless of the technique used to provide symbols, the models trained in the presence of symbols performed significantly better than those trained in the absence of symbols. We did not find much difference between the models trained with the explicit symbols and those trained by learning to classify the input.

The second objective was to explain the role of symbols. In Section \ref{sec:theory-approach}, we expanded the objective by stating that in order to explain why symbols improve the accuracy of trained models, we have to show that the presence of symbols allows the neural networks to discover a representation that captures an algorithm that performs arithmetic operations, just as with humans symbols allow learners to generalize to unseen concepts. Experiments 4 and 5 in Section \ref{sec:empirical-studies-explaining-the-role-of-symbols} showed that this is a difficult task for networks trained using one-hot vector symbols. We were however able to show that the networks trained with symbols were able to capture the carry forward process of addition. This led us to consider other representations for the symbols.

We theorized that a symbol must exhibit three characteristics in order to be able to guide a learner to a general solution. They must be able to represent quantity, ordinal relations and capture the operation being performed. By using temperature encoded symbols we were able to develop LSTM based recurrent neural networks that perform well on combinations of digits that were not seen during training. We therefore concluded that temperature encoded symbols are able to capture the three aspects needed by the symbols to discover an algorithm that can perform the arithmetic operation, thereby accomplishing the second objective. 

\section{Research Contributions} \label{sec:findings-and-conclusion-research-contributions}

This section lists contributions from our research that we believe to be novel and of significant interest.

\begin{itemize}
	\item Symbols that are presented on the network's output are as effective as those presented as an input to the neural network. Experiment 1 in Section \ref{sec:experiment-1} showed that the mean accuracy of the models trained with explicit symbols (NS), implicit symbols (NC) and both (NX) are statistically similar.
	\item The encoding used to represent symbols has a significant impact on the utility of the symbols and the overall effectiveness of the model. When we contrast the results obtained from Experiment 4 in Section \ref{sec:experiment-4} (one-hot vector symbols) with the ones obtained from Experiment 7 in Section \ref{sec:experiment-7} (temperature encoded symbols), not only do the models trained using temperature encoded symbols show significant improvement on the unseen test set, they also show slight improvement on the seen test set.
	\item Training neural networks with the aid of symbols takes more time than without the aid of symbols, contrary to our expectations. In Section \ref{sec:experiment-3}, we attributed this outcome to the possibility that it takes longer for the learning algorithm to find a suitable set of weights that fits both the symbolic data and the noisy handwritten digits.
\end{itemize}

\section{Future Work} \label{sec:findings-and-conclusion-future-work}

This final section suggests potential avenues for future development.

\begin{itemize}
	\item We focused our experimental efforts on training recurrent neural networks in the presence of symbols to learn arithmetic using images of handwritten digits. Future work can consider applying this theory of learning with symbols to other application domains. Specifically, natural language processing or medical imagery.
	\item Further qualitative investigations can be made similar to the ones presented in Experiment 5 in Section \ref{sec:experiment-5} and Experiment 6 in \ref{sec:experiment-6} to understand the development of algorithms in the neural network representation for operators other than addition.
	\item Our experiments used sequences of two operands. More experiments can be done to investigate the theory on longer sequences of operands. Specifically, experiments can be conducted to understand how the network would behave when the carry signal for addition would have to be carried over more than one place. Our current setup of using reverse Polish notation, although was not necessary for the experiments presented here, should eliminate the need for taking operator precedence into account.
	\item Instead of using recurrent neural networks and adding a symbolic channel, a feed forward neural network trained using a context sensitive multi-task learning (csMTL) approach like the one presented by Poirier and Silver\cite{silver2007context} could be investigated. The context input can be used to configure the network to either perform classification on the inputs or perform the operation. Our belief is the symbolic representation would be acquired implicitly by the network by learning to classify the operands within the one csMTL network.
\end{itemize}