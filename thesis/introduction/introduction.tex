\chapter{Introduction} \label{sec:introduction}

Imagine you are lost in a jungle, wandering alone in an unfamiliar environment fraught with danger and uncertainty. Your survival depends on your ability to learn to differentiate between the things you can eat and the things that can eat you. This is an expensive and risky process given the need for movement and trial and error. One day you come across another individual that is more familiar with your surroundings. Together you develop a common sign language (symbols) that you can use to know more about the threats and opportunities in this hostile environment. Along with real examples, these symbols allow you to overcome your struggle in the jungle.

Learning in humans involves being introduced to many examples. The more examples the learner is exposed to the more effective the learning process is. This however is not always possible given the cost (energy and time) and risk (injury and death) involved. The ability of a single learner to learn effectively is therefore hampered by the lack of a sufficient number of examples. However, societies evolved to overcome this challenge by developing mechanisms by which accumulated knowledge is shared amongst individuals in the group\cite{DBLP:journals/corr/abs-1203-2990}.

Similar to how the lack of a sufficient number of examples can make learning difficult for humans, machine learning models also experience difficulty in finding a suitable representation when they are trained with a limited dataset. The research presented in this thesis draws analogies between training neural network models and learning in humans.  We investigate how the presence of symbols while training artificial neural networks with impoverished datasets can improve the effectiveness of these models, just like the exchange of symbols between individual human learners can help them overcome their learning challenges.

We begin this chapter by providing a brief introduction to machine learning and specifically deep learning followed by an overview of learning with symbols. We then discuss our research scope and objective. Finally, we present an outline to the remainder of the thesis.

\section{Machine Learning with Neural Networks} \label{sec:introduction-introduction-machine-learning}

Designing computer systems that can model the world in such a way as to exhibit what we call intelligence requires such systems to have the ability to interpret a large number of factors. For example, if we were to develop a system that can identify and perform calculations using numbers from images of handwritten digits, the system would have to account for all the possible variations in which each digit can be written. Given the amount of data required, it would be infeasible to manually formalize solutions to these problems. Machine Learning algorithms can instead be used to automate the development of a solution by learning the important features that represent the problem domain and formalize them in a way that can then be applied to different instances of the problem\cite{Bengio:2009:LDA:1658423.1658424}.

\subsection{Neural Networks} \label{sec:introduction-introduction-machine-learning-neural-networks}

Many advances in artificial intelligence (AI) research are inspired by work done in other fields, specifically those that aim to study human cognition, language and social interaction\cite{DBLP:journals/corr/abs-1203-2990}. Deep learning, which is a form of machine learning, is one particular area of AI that is heavily influenced by how the human brain works.

In deep learning, models are constructed to represent tasks (problems) using processing units called artificial neurons that are inspired by how biological neurons function in living organisms. These neurons are connected in layers to form artificial neural networks, that when properly trained, have the power to represent solutions to these complex problems. Section \ref{sec:background-artificial-neural-networks} provides a detailed description of artificial neural networks.

When training neural network models, an appropriate set of weights is discovered that allows the network to capture the factors of variation exhibited in the problem and therefore approximate an unknown function $f$. The training process requires a set of training examples that are applied iteratively to the model. After each iteration, the error produced by the model is calculated and the model parameters are updated in such a way as to attempt to reduce that error in subsequent iterations. The goal of the learning algorithm is to discover the approximating function $\hat{f}$ that will accurately generalize to yet unseen examples.


\subsection{Challenges in Training Neural Networks} \label{sec:introduction-introduction-machine-learning-challenges-training-neural-networks}

There are two factors that affect the effectiveness (accuracy) of a trained model, namely the variance exhibited in the training examples and the bias assumed in the model. Variance refers to how much the model errs because of changes in the training data, whereas the bias of a model refers to how much the model errs because of incorrect assumptions made by the learning algorithm approximating $\hat{f}$ when compared to the real function $f$. An ideal model will exhibit both low variance and low bias\cite{James:2014:ISL:2517747}.

The variance factor tells us that the size and quality of our training dataset has a huge impact on the quality of the models being developed. The larger the training set and greater the algorithm's representation power, the more accurate our models will be\cite{James:2014:ISL:2517747}. The next section introduces a theory, that including symbols of noisy concepts helps overcome the challenges of training neural networks from small impoverished datasets. Symbols provide concise and accurate information to the learning algorithm that can assist in forming more accurate models.

\section{Learning with Symbols} \label{sec:introduction-learning-symbols}

Humans generally don't need as many training examples to learn a specific concept or task compared to machine learning systems. For example, a human learner can learn to identify an object in an image using only a single example. When we contrast that with an artificial neural network, the neural network might need hundreds of examples to achieve the same level of accuracy. We as humans build on the knowledge we've accumulated over our lifetimes\cite{Thrun1998}. Designing deep learning systems that accumulate and integrate knowledge over time and over multiple tasks is therefore important to overcome the challenge of learning from impoverished datasets\cite{silver2013lifelong}.

When considering how learning in humans works, it is believed that the ability of individual humans to learn new concepts accurately is hampered by the number and variety of examples of the concepts they are each exposed to. However, this difficulty is overcome when individuals communicate concise examples of concepts between each other using a common language\cite{DBLP:journals/corr/abs-1203-2990}. This common language, which we refer to in this thesis as symbols, is very effective at transferring the knowledge, that has accumulated in a society over a long period of time, from one generation to the next.

Deep artificial neural networks experience similar challenges when learning new concepts from limited datasets. This can be attributed to the existence of many local minima in the hypothesis space of the network's loss function\cite{Larochelle:2009:EST:1577069.1577070}. We therefore hypothesize that similar to how human learners use symbols to overcome difficulty in learning from limited real examples, artificial neural networks can also benefit from symbols to improve the effectiveness (accuracy) of training using a limited dataset.

The research presented in this thesis empirically investigates the effect of introducing a clear and consistent symbolic channel on the accuracy of neural network models. We propose that the clear symbols introduced during training introduce inductive bias to the network which guides the training algorithm to a more optimal representation\cite{Thrun1998}. The next section states the objective and scope of our research.

\section{Research Objective and Scope} \label{sec:introduction-research-objective-scope}

Our goal is to draw parallels between our understanding of how humans learn with symbols and how learning with symbols can best be applied to machine learning algorithms. The objective is to demonstrate that providing clear and concise knowledge in the form of symbols to a deep neural network can improve the effectiveness of the training process by significantly reducing the number of noisy training examples needed to produce a high level of accuracy.

\subsection{Research Objective} \label{sec:introduction-research-objective-scope-research-objective}

Knowledge sharing will be empirically demonstrated in this thesis by training deep neural networks to perform basic arithmetic operations (such as addition, subtraction, multiplication and division) on images of handwritten digits. These models will accept the images as inputs and will output the result of the operation as vectors that encode the value of the output. The models will also accept a symbolic channel that emulates shared knowledge between learning agents. Training will be performed with and without the presence of symbols. The results will then be analyzed to show the impact of the symbolic channel on the accuracy of the models developed.

Besides showing that the presence of symbols increases the accuracy of artificial neural networks, we also want to empirically explain why symbols allow for this improvement. In deep neural networks, layers of the architecture learn abstractions that can be shared among different tasks\cite{Bengio:2009:LDA:1658423.1658424}. We believe that some of these abstractions can be learned more effectively from clear symbols than they can from noisy inputs. This leads to better models than if the networks were trained using noisy handwritten digits alone.

\subsection{Scope} \label{sec:introduction-research-objective-scope-scope}

We focus our research on understanding the effect of symbols on the accuracy of deep neural networks, we also examine the effectiveness of different approaches to encode and provide symbols to our neural network models. Several models are developed to investigate the effects of symbols on training accuracy and to a lesser extent on the speed of training. This thesis presents the results obtained from these experiments as well as an analysis of these results. We limit our experiments to developing neural network architectures using recurrent neural networks based on Long Short Term Memory units (LSTMs). The application domain is also limited to training the LSTM networks to learn to perform addition, subtraction, multiplication and division on images of handwritten digits from the MNIST dataset (see Section \ref{sec:theory-approach-the-mnist-dataset}).

Other types of neural network architectures, such as Convolutional Neural Networks, will not be considered for this problem. MNIST images are preprocessed so that all digits are centered and are of the same orientation and scale. It has been shown that traditional feed-forward networks can perform very well on the MNIST dataset without the need for convolution. Therefore, in order to reduce complexity and focus on the problem at hand, convolutional neural networks are not used. 

\section{Thesis Outline} \label{sec:introduction-thesis-outline}

This chapter introduced the basis for our theory and the goal and strategy of our research. The following is an outline of the remaining chapters in this thesis.

Chapter 2 introduces the background knowledge that is required by the reader in order to understand the remainder of the thesis. We start Chapter 2 by introducing artificial neural networks (Section \ref{sec:background-artificial-neural-networks}), including a discussion on deep learning. Next, in Section \ref{sec:background-sequence-modeling}, a detailed explanation of recurrent neural networks (RNNs) and long short-term memory units (LSTMs) is presented. These models form an important aspect of our empirical studies. Finally, Section \ref{sec:background-problem-local-minima} wraps up Chapter 2 by introducing the core problem that our theory is based on, that of local minima.

Chapter 3 formally lays out the hypothesis that forms the basis for our research. This is followed by an explanation of our theory development and the approach being proposed to investigate and validate this theory. 

Chapter 4 presents the empirical studies used to test and further develop the theory. Each experiment performed is detailed along with the results and a discussion of those results.

Chapter 5 is the last chapter and provides a summary of our findings and contributions. We also suggest avenues for future work.