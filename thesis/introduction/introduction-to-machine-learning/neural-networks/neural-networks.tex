\subsection{Neural Networks} \label{sec:introduction-introduction-machine-learning-neural-networks}

Many advances in artificial intelligence (AI) research are inspired by work done in other fields, specifically those that aim to study human cognition, language and social interaction\cite{DBLP:journals/corr/abs-1203-2990}. Deep learning, which is a form of machine learning, is one particular area of AI that is heavily influenced by how the human brain works.

In deep learning, models are constructed to represent tasks (problems) using processing units called artificial neurons that are inspired by how biological neurons function in living organisms. These neurons are connected in layers to form artificial neural networks, that when properly trained, have the power to represent solutions to these complex problems. Section \ref{sec:background-artificial-neural-networks} provides a detailed description of artificial neural networks.

When training neural network models, an appropriate set of weights is discovered that allows the network to capture the factors of variation exhibited in the problem and therefore approximate an unknown function $f$. The training process requires a set of training examples that are applied iteratively to the model. After each iteration, the error produced by the model is calculated and the model parameters are updated in such a way as to attempt to reduce that error in subsequent iterations. The goal of the learning algorithm is to discover the approximating function $\hat{f}$ that will accurately generalize to yet unseen examples.
